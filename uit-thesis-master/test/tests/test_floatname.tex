\documentclass[nocopyright,norsk,USenglish]{uit-thesis-test}

\floatname{figure}{New Figure Name}

\newcommand{\definitionname}{DEF}
\newcommand{\listdefinitionname}{My list of definitions}
\newlistof{definition}{def}{\listdefinitionname}
\newcommand{\definition}[1]{%
  \refstepcounter{definition}%
  \par\noindent\textbf{\definitionname~\thedefinition. #1}%
  \addcontentsline{def}{definition}
    {\protect\numberline{\thechapter.\thedefinition}#1}\par%
}
\floatname{definition}{First Definition Name}

\newfloat{code}{!tbph}{locode}[chapter]
\newlistof{code}{locode}{List of Code Samples}

\begin{document}

\frontmatter
\tableofcontents
\listofdefinition
\listofcode

\mainmatter

\chapter{A chapter}

\makeatletter
Figure name: `\figurename'\\
Figure float name: `\@nameuse{fname@figure}'\\
Table name: `\tablename'\\
Table float name: `\@nameuse{fname@table}'\\
Definition name: `\definitionname'\\
Definition float name: `\@nameuse{fname@definition}'\\
Code name: `\codename'\\
Code float name: `\@nameuse{fname@code}'
\makeatother

Look at the example in \autoref{fig:a}.

\begin{figure}[!htbp]
\centering
\fbox{FIRST EXAMPLE FIGURE}
\caption{Some figure text}\label{fig:a}
\end{figure}

\floatname{figure}{Redefined Figure Name}

\makeatletter
Figure name: `\figurename'\\
Figure float name: `\@nameuse{fname@figure}'\\
Table name: `\tablename'\\
Table float name: `\@nameuse{fname@table}'\\
Definition name: `\definitionname'\\
Definition float name: `\@nameuse{fname@definition}'\\
Code name: `\codename'\\
Code float name: `\@nameuse{fname@code}'
\makeatother

Now, look at \autoref{fig:b}.

\begin{figure}[!htbp]
\centering
\fbox{SECOND EXAMPLE FIGURE}
\caption{Some figure text}\label{fig:b}
\end{figure}

Here we have a definition:

\definition{A definition}\label{def:a}

And another:

\definition{Some other definition}\label{def:b}

Remember, from \autoref{def:a}, that something\ldots

\floatname{definition}{Redefined Definition Name}

Remember, from \autoref{def:b}, that something else\ldots

\definition{Yet another definition}


Here is \autoref{code:a}:

\begin{code}
\centering
\fbox{This is an algorithm}
\caption{An algorithm}\label{code:a}
\end{code}

\floatname{code}{Redefined Code Name}

And here is \autoref{code:b}:

\begin{code}
\centering
\fbox{This is another algorithm}
\caption{Another algorithm}\label{code:b}
\end{code}


\captionsnorsk
\extrasnorsk

Now we have changed language to Norwegian.

And we have \autoref{fig:a} and \autoref{def:a} and \autoref{code:a}.

\begin{figure}[!htbp]
\centering
\fbox{THIRD EXAMPLE FIGURE}
\caption{Some figure text}\label{fig:c}
\end{figure}

And

\definition{More definitions}

And

\begin{code}
\centering
\fbox{Code, code, code}
\caption{Even more algorithms}\label{code:c}
\end{code}

\makeatletter
Figure name: `\figurename'\\
Figure float name: `\@nameuse{fname@figure}'\\
Table name: `\tablename'\\
Table float name: `\@nameuse{fname@table}'\\
Definition name: `\definitionname'\\
Definition float name: `\@nameuse{fname@definition}'\\
Code name: `\codename'\\
Code float name: `\@nameuse{fname@code}'
\makeatother

\captionsUSenglish
\extrasUSenglish

Now, we have switched back to US English.

And we have \autoref{fig:a} and \autoref{def:a} and \autoref{code:a}.

\begin{figure}[!htbp]
\centering
\fbox{FOURTH EXAMPLE FIGURE}
\caption{Some figure text}\label{fig:d}
\end{figure}

And

\definition{More definitions}

And

\begin{code}
\centering
\fbox{Code, code, code}
\caption{Even more algorithms}\label{code:d}
\end{code}

\makeatletter
Figure name: `\figurename'\\
Figure float name: `\@nameuse{fname@figure}'\\
Table name: `\tablename'\\
Table float name: `\@nameuse{fname@table}'\\
Definition name: `\definitionname'\\
Definition float name: `\@nameuse{fname@definition}'\\
Code name: `\codename'\\
Code float name: `\@nameuse{fname@code}'
\makeatother

\end{document}
